
\section{方法设计}

\subsection{方法推导}

在三阶的Halley方法的基础上再进行一次牛顿法迭代得到的序列,显然是六阶收敛的:
\begin{equation}
    \label{eq:halley}
    \begin{cases}
        z_n &= x_n - \dfrac{2f(x_n)f'(x_n)}{2[f'(x_n)]^2-f(x_n)f''(x_n)}, \\
        x_{n+1} &= x_n - \dfrac{f(z_n)}{f'(z_n)}   
    \end{cases}
\end{equation}

参考\cite{eftekhari2014new}中的方法,希望消去$\{z_n\}_{n=0}^\infty$中的二阶导数,如果记$y_n = x_n - \dfrac{f(x)}{f'(x)}$,则有$\lim\limits_{n\to \infty}\dfrac{y_n}{x_n} = 1$,因此二阶导数可以使用一阶导计算前向差商近似为:
\begin{equation}
    \label{eq:second-derivative}
    \begin{aligned}
        f''(x_n) &= \lim\limits_{n\to \infty}\dfrac{f'(y_n) - f'(x_n)}{y_n - x_n}\\ 
        &\approx \dfrac{f'(y_n) - f'(x_n)}{y_n - x_n},
    \end{aligned}
\end{equation}
将公式\ref{eq:second-derivative}代入公式\ref{eq:halley},使用sympy\footnote{sympy是Python的符号计算库,https://www.sympy.org/en/index.html}(代码见附录1)化简得:
\begin{equation}
    \label{eq:my}
    \begin{cases}
        y_n &= x_n - \dfrac{f(x)}{f'(x)} \\
        z_n &= x_n - \dfrac{2f(x_n)}{f'(x_n) + f'(y_n)} \\
        x_{n+1} &= z_n - \dfrac{f(z_n)}{f'(z_n)}
    \end{cases}
\end{equation}
公式\ref{eq:my}即为本文所提出的方法,由于它使用了差商代替二阶导数,因此需要对收敛阶进行证明。

\subsection{收敛阶证明}

我参考了\cite{kou2006modified}和\cite{eftekhari2014new}使用的技巧对公式\ref{eq:my}的收敛阶进行证明。由于涉及到\textbf{6阶泰勒展开}的计算较为复杂,因此我使用了sympy\footnotemark[1]进行符号计算和化简,\textbf{代码在附录2中,本部分所推导的结果皆由该代码产生,其正确性保证了本文推导结果的正确}。由于篇幅有限,部分泰勒展开用省略号表示。

假设函数$f(x)$的单根是$\alpha$且$f'(\alpha)\neq 0$,记截断误差$e_n = \alpha - x_n$,对$f(x_n)$在$\alpha$处进行泰勒展开得:
\begin{equation}
    \label{eq:fx-taylor}
    \begin{aligned}
        f(x_n) &= f(\alpha + e_n) \\
        &= f(\alpha) + f'(\alpha)e_n + \dfrac{1}{2}f''(\alpha)e_n^2+\dfrac{1}{3!}f'''(\alpha)e_n^3+ \cdots + O(e_n^7) \\
        &= f'(\alpha)\left[e_n + \dfrac{f''(\alpha)e_n^2}{2f'(\alpha)}+\dfrac{f'''(\alpha)e_n^3}{3!f'(\alpha)} + \cdots + O(e_n^7) \right],
    \end{aligned}
\end{equation}
注意到$f(\alpha)=0$,所以公式中将它消去了,记$c_k=\dfrac{f^{(k)}(\alpha)}{f'(\alpha)k!}$,则有
\begin{equation}
    \begin{aligned}
        f(x_n) &= f'(\alpha)\left[c_1e_n + c_2e_n^2 + c_3e_n^3 + c_4e_n^4 + c_5e_n^5 + c_6e_n^6 + O(e_n^7)\right] \\
        &= f'(\alpha)\left[ \sum\limits_{k=1}^6 c_ke_n^k + O(e_n^7) \right]
    \end{aligned}
\end{equation}
同理对$f'(x_n)$在$\alpha$处进行泰勒展开得:
\begin{equation}
    \label{eq:dx}
    \begin{aligned}
        f'(x_n) &= f'(\alpha + e_n)\\
        &= f'(\alpha) + f''(\alpha)e_n + \dfrac{1}{2}f'''(\alpha)e_n^2 + \cdots + O(e_n^6) \\
        &=f'(\alpha) [1 + 2c_2e_n + 3c_3e_n^2 + 4c_4e_n^3 + 5c_5e_n^4 + 6c_6e_n^6 + O(e_n^6)] \\
        &= f'(\alpha) \left[ 1 + \sum\limits_{k=2}^6 kc_ke_n^{k-1} + O(e_n^6) \right]
    \end{aligned}
\end{equation}
希望得到$y_n$的泰勒展开形式,因此首先需要计算:
\begin{equation}
    \label{eq:fx-div-dx}
    \dfrac{f(x_n)}{f'(x_n)} = \dfrac{\sum\limits_{k=1}^6 c_ke_n^k + O(e_n^7)}{1 + \sum\limits_{k=2}^6 kc_ke_n^{k-1} + O(e_n^6) },
\end{equation}
记$\theta = 1 + \sum\limits_{k=2}^6 kc_ke_n^{k-1} + O(e_n^6) $,显然$\lim\limits_{n\to \infty}\theta = 0$,因此在$\theta = 0$处对分母进行泰勒展开有:
\begin{equation}
    \label{eq:fenmu-taylor}
    \begin{aligned}
        \dfrac{1}{1+\theta} &= 1 - \theta - 2 c_{2} e + \theta^2 - \theta^3 + \theta^4 - \theta^5 + O(\theta^6) \\
        &= 1 + e_n^{2} \cdot \left(4 c_{2}^{2} - 3 c_{3}\right) + e_n^{3} \left(- 8 c_{2}^{3} + 12 c_{2} c_{3} - 4 c_{4}\right) \\
        &\qquad + e_n^{4} \cdot \left(16 c_{2}^{4} - 36 c_{2}^{2} c_{3} + 16 c_{2} c_{4} + 9 c_{3}^{2} - 5 c_{5}\right) \\
        &\qquad + e_n^{5} \left(- 32 c_{2}^{5} + 96 c_{2}^{3} c_{3} - 48 c_{2}^{2} c_{4} - 54 c_{2} c_{3}^{2} + 20 c_{2} c_{5} + 24 c_{3} c_{4} - 6 c_{6}\right) + O\left(e_n^{6}\right),
    \end{aligned}
\end{equation}
将公式\ref{eq:fenmu-taylor}代入公式\ref{eq:fx-div-dx}并化简,由于化简过程较为复杂,为了保证准确性,再次使用sympy进行验证(代码见附录2):
\begin{equation}
    \begin{aligned}
        \dfrac{f(x_n)}{f'(x_n)} &= e - c_{2} e_n^{2} + e_n^{3} \cdot \left(2 c_{2}^{2} - 2 c_{3}\right) + e_n^{4} \cdot \left(- 4 c_{2}^{3} + 7 c_{2} c_{3} - 3 c_{4}\right) \\
        &\qquad + e_n^{5} \cdot \left(8 c_{2}^{4} - 20 c_{2}^{2} c_{3} + 10 c_{2} c_{4} + 6 c_{3}^{2} - 4 c_{5}\right) \\
        &\qquad + e_n^{6} \cdot \left(- 16 c_{2}^{5} + 52 c_{2}^{3} c_{3} - 28 c_{2}^{2} c_{4} - 33 c_{2} c_{3}^{2} + 13 c_{2} c_{5} + 17 c_{3} c_{4} - 5 c_{6}\right) + O\left(e_n^{7}\right),
    \end{aligned}
\end{equation}
因此求得$y_n$的泰勒展开:
\begin{equation}
    \label{eq:taylor-y}
    \begin{aligned}
        y_n &= x_n - \dfrac{f(x_n)}{f'(x_n)}\\
        &= \alpha + c_{2} e_n^{2} + e_n^{3} \cdot \left(- 2 c_{2}^{2} + 2 c_{3}\right) + e_n^{4} \cdot \left(4 c_{2}^{3} - 7 c_{2} c_{3} + 3 c_{4}\right) \\
        &\qquad + e_n^{5} \cdot \left(- 8 c_{2}^{4} + 20 c_{2}^{2} c_{3} - 10 c_{2} c_{4} - 6 c_{3}^{2} + 4 c_{5}\right) \\
        &\qquad + e_n^{6} \cdot \left(16 c_{2}^{5} - 52 c_{2}^{3} c_{3} + 28 c_{2}^{2} c_{4} + 33 c_{2} c_{3}^{2} - 13 c_{2} c_{5} - 17 c_{3} c_{4} + 5 c_{6}\right) + O\left(e_n^{7}\right),
    \end{aligned}
\end{equation}
\textbf{公式\ref{eq:taylor-y}同时也证明了牛顿法是二阶收敛的},截断误差为$\dfrac{f''(\alpha)}{2f'(\alpha)}e_n^2 + O(e_n^3)$。继续对$f'(y_n)$在$\alpha$处做泰勒展开有:
\begin{equation}
    \label{eq:dy}
    \begin{aligned}
        f'(y_n) &= f'(\alpha) + [c_2e_n^2 + (2c_2^2 - 2c_3)e_n^3 + O(e_n^4)]f''(\alpha) + \cdots + O(e_n^6) \\
        &= 1 + 2 c_{2}^{2} e_n^{2} + e_n^{3} \left(- 4 c_{2}^{3} + 4 c_{2} c_{3}\right) + e_n^{4} \cdot \left(8 c_{2}^{4} - 11 c_{2}^{2} c_{3} + 6 c_{2} c_{4}\right) \\
        &\qquad + e_n^{5} \cdot \left(- 16 c_{2}^{5} + 28 c_{2}^{3} c_{3} - 20 c_{2}^{2} c_{4} + 8 c_{2} c_{5}\right) + O\left(e_n^{6}\right),
    \end{aligned}
\end{equation}
将公式\ref{eq:dx}与公式\ref{eq:dy}相加得到$z_n$的分母部分的泰勒展开:
\begin{equation}
    \label{eq:zn-fenmu}
    f'(x_n) + f'(y_n) = 2f'(\alpha)\left[ 1 + c_2e_n + \left( c_2^2 + \dfrac{3}{2}c_3 \right)e_n^2 + \cdots + O(e_n^6) \right],
\end{equation}
因此将公式\ref{eq:zn-fenmu}与公式\ref{eq:fx-taylor}相除得到$z_n$的泰勒展开:
\begin{equation}
    \begin{aligned}
        x_n - \dfrac{2f(x_n)}{f'(x_n) + f'(y_n)} &= \alpha + e_n - \dfrac{2\left[ \sum\limits_{k=1}^6 c_ke_n^k + O(e_n^7) \right]}{1 + c_2e_n+ \left(c_2^2 + \dfrac{3}{2}c_3  \right)e_n^2 + \cdots + O(e_n^6)},
    \end{aligned}
\end{equation}
使用与公式\ref{eq:fenmu-taylor}相似的技巧,对分母部分进行泰勒展开得:
\begin{equation}
    \label{eq:taylor-z}
    \begin{aligned}
        x_n - \dfrac{2f(x_n)}{f'(x_n) + f'(y_n)} &= \alpha + e_n^{3} \cdot \left(c_{2}^{2} + \frac{c_{3}}{2}\right) + e_n^{4} \cdot \left(- 3 c_{2}^{3} + \frac{3 c_{2} c_{3}}{2} + c_{4}\right) \\
        &\qquad + e_n^{5} \cdot \left(6 c_{2}^{4} - 9 c_{2}^{2} c_{3} + 2 c_{2} c_{4} - \frac{3 c_{3}^{2}}{4} + \frac{3 c_{5}}{2}\right) \\
        &\qquad + e_n^{6} \cdot \left(- 9 c_{2}^{5} + 25 c_{2}^{3} c_{3} - 15 c_{2}^{2} c_{4} - \frac{5 c_{2} c_{3}^{2}}{2} + \frac{5 c_{2} c_{5}}{2} - \frac{5 c_{3} c_{4}}{2} + 2 c_{6}\right) \\
        &\qquad + O\left(e_n^{7}\right),
    \end{aligned}
\end{equation}
\textbf{公式\ref{eq:taylor-z}同时也证明了序列$\{z_n\}_{n=0}^\infty$是三阶收敛的},即使用一阶导数的前向差商来计算二阶导数的Halley方法是仍是三阶收敛的。继续计算$f(z_n)$和$f'(z_n)$的泰勒展开,使用与公式\ref{eq:fx-taylor}-\ref{eq:taylor-y}相同的方法,可以得到$x_{n+1}$的泰勒展开:
\begin{equation}
    \label{eq:taylor-newx}
    x_{n+1} = \alpha + e^{6} \cdot \left(c_{2}^{5} + c_{2}^{3} c_{3} + \frac{c_{2} c_{3}^{2}}{4}\right) + O\left(e^{7}\right),
\end{equation}
因此证明了本方法是六阶收敛的,截断误差为$e^{6} \cdot \left(c_{2}^{5} + c_{2}^{3} c_{3} + \frac{c_{2} c_{3}^{2}}{4}\right) + O\left(e^{7}\right)$


\subsection{一些有趣的发现}

由于使用sympy可以很方便地进行泰勒展开来对收敛阶进行推导,我又尝试了一下修改迭代格式,使用与$\{z_n\}_{n=0}^\infty$相似的格式来代替最后一步的牛顿法:
\begin{equation}
    \label{eq:my-interesting}
    \begin{cases}
        y_n &= x_n - \dfrac{f(x)}{f'(x)} \\
        z_n &= x_n - \dfrac{2f(x_n)}{f'(x_n) + f'(y_n)} \\
        x_{n+1} &= z_n - \dfrac{2f(z_n)}{f'(y_n) + f'(z_n)}
    \end{cases},
\end{equation}
附录2的最后几行代码展示了计算过程,与公式\ref{eq:dy}-\ref{eq:taylor-newx}类似,可以得到$x_{n+1}$的泰勒展开:
\begin{equation}
    e^{5} \cdot \left(c_{2}^{4} + \frac{c_{2}^{2} c_{3}}{2}\right) + e^{6} \cdot \left(- 5 c_{2}^{5} + \frac{5 c_{2}^{3} c_{3}}{2} + c_{2}^{2} c_{4} + c_{2} c_{3}^{2}\right) + \alpha + O\left(e^{7}\right),
\end{equation}
\textbf{因此新得到的公式\ref{eq:my-interesting}只有5阶收敛},但它的计算量比牛顿法还多(同样需要计算$f(z_n)$和$f'(z_n)$,且需要额外计算一次加法和乘法),进一步让我感受到了牛顿法无可比拟的优势。




