
\section*{摘要}

本文以张老师上课所介绍的构造8阶收敛的牛顿法,和PPT第63页简要介绍Steffenson和Halley方法为基础和启发,构造了6阶收敛的非线性方程组求根方法,并使用前向差商代替二阶导数从而实现只需求一阶导数,并使用泰勒展开证明了其收敛阶。8个函数的数值试验充分对比了本方法与牛顿法、Halley法和两种现有的6阶方法,并对重根情况进行了扩展讨论。结果显示本方法具有较好的数值稳定性。


