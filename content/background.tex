
\section{引言}

本节主要对张老师上课所介绍的构造8阶收敛的牛顿法,和PPT第63页简要介绍Steffensen和Halley方法做一个简要的回顾,作为本文的基础。

\subsection{8阶收敛的牛顿法}

使用三次牛顿法反复代入,如公式\ref{eq:eight-newton}所示,可以得到序列$\{y_n\}_{n=0^\infty}$是2阶收敛,$\{z_n\}_{n=0^\infty}$是4阶收敛,因此最终$\{x_n\}_{n=0}^{\infty}$是8阶收敛的,这是一种简单的提高迭代公式阶数的方法。
\begin{equation}
    \label{eq:eight-newton}
    \begin{aligned}
        y_n = x_n - \dfrac{f(x_n)}{f'(x_n)}\\
        z_n = y_n - \dfrac{f(y_n)}{f'(y_n)}\\
        x_{n+1} = z_n - \dfrac{f(z_n)}{f'(z_n)}
    \end{aligned}
\end{equation}


\subsection{Steffenson方法}

由于在零点附近$\lim\limits_{n\to \infty}f(x_n)=0$,因此可以使用前向差商可以将零点附近的导数近似为:
\begin{equation}
    \label{eq:approx-derivative}
    f'(x)\approx \dfrac{f(x_n+f(x_n))}{f(x_n)}
\end{equation}

将公式\ref{eq:approx-derivative}代入牛顿迭代公式,可以得到:
\begin{equation}
    \label{eq:newton-nodev}
    x_{n+1}=x_n-\dfrac{f(x_n)^2}{f(x+f(x_n))-f(x_n)}
\end{equation}

这就是PPT第63页所提到的Steffensen公式,它是也二阶收敛的\cite{jain2007steffensen},同时避免了求导。

\subsection{Halley方法}

PPT第63页同样提到了三阶收敛的Halley方法和Chebyshev方法,查阅文献\cite{gutierrez1997family}后发现,它们属于同一族方法,具有如下形式:
\begin{equation}
    x_{n+1} = x_n-\left( 1 + \dfrac{1}{2}\dfrac{L_f(x_n)}{1-\beta L_f(x_n)} \right) \dfrac{f(x_n)}{f'(x_n)},
\end{equation}
其中
\begin{equation}
    L_f(x_n) = \dfrac{1}{2}\dfrac{f''(x_n)f(x_n)}{[f'(x_n)]^2}
\end{equation}

特别的,当$\beta=0$时为Chebyshev方法,$\beta=\frac{1}{2}$时为Halley方法,$\beta=1$时为超Halley方法。不少研究对这族方法进行改进,但是它们都具有\textbf{较为复杂的迭代格式},如\cite{kou2006modified}提出的一种无三阶收敛的方法:
\begin{equation}
    \begin{cases}
        y_n &= x_n - \theta\dfrac{f(x_n)}{f'(x_n)}, \\
        x_{n+1} &= x_n - \dfrac{\theta^2 f(x_n)}{(\theta^2 - \theta + 1)f(x_n) - f(y_n)}\dfrac{f(x_n)}{f'(x_n)},\quad (\theta \in \mathbb{R}, \theta \neq 0),
    \end{cases}
\end{equation}
又如\cite{chun2007some}提出的另一种三阶收敛的方法:
\begin{equation}
    \begin{cases}
        w_n & = x_n - \dfrac{f(x_n)}{f'(x_n)}, \\
        K_\alpha(x_n) & = \dfrac{2f(x_n)f(w_n)(1+\alpha f'^2(x_n))}{f^2(x_n)+\alpha f'^2(x_n)[f(w_n)-f(x_n)]^2}, \quad (\alpha \in \mathbb{R}), \\
        x_n+1 & = x_n - \left(1+\dfrac{1}{2}\dfrac{K_\alpha(x_n)}{1-\beta K_\alpha(x_n)}\right) \dfrac{f(x_n)}{f'(x_n)}, \quad (\beta \in \mathbb{R}),
    \end{cases}
\end{equation}


这些方法与同为三阶收敛的其它方法相比,或多或少能减少一些迭代次数,但代价是较为复杂的迭代格式。

\subsection{六阶收敛的方法}

在三阶收敛方法的基础上,再使用一次牛顿迭代或其它二阶迭代法,即可实现六阶收敛的方法。如开山鼻祖\cite{neta1979sixth}中提出的六阶收敛的一族方法:
\begin{equation}
    \begin{cases}
        w_n &= x_n - \dfrac{f(x_n)}{f'(x_n)}, \\
        z_n &= w_n - \dfrac{f(w_n)}{f'(x_n)}\dfrac{f(x_n)+\beta f(w_n)}{f(x_n) + (\beta - 2)f(w_n)}, \\
        x_{n+1} &= z_n - \dfrac{f(w_n)}{f'(x_n)}\dfrac{f(x_n)-f(w_n)+\gamma f(z_n)}{f(x_n) - 3f(w_n) + \lambda f(z_n)}
    \end{cases}
\end{equation}
又如\cite{grau2006improvement}中对Ostrowski方法\cite{householder1967solution}改进提出的六阶收敛的方法:
\begin{equation}
    \begin{cases}
        y_n &= x_n - \dfrac{f(x_n)}{f'(x_n)}, \\
        z_n &= y_n - \dfrac{y_n - x_n}{2f(y_n) - f(x_n)}f(y_n), \\
        x_{n+1} &= z_n - \dfrac{y_n - x_n}{2f(y_n) - f(x_n)}f(z_n),
    \end{cases}
\end{equation}
又如\cite{soleymani2011new}中提出的用前向差商代替导数的六阶收敛的方法:
\begin{equation}
    \begin{cases}
        w_n &= x_n + f(x_n), \\
        y_n &= x_n - \dfrac{f(x_n)}{f[x_n, w_n]}, \\
        z_n &= x_n - \dfrac{f(x_n)}{f[x_n, w_n]}\left( 1 + \dfrac{f(y_n)}{f(x_n)} \left( 1 + 2\dfrac{f(y_n)}{f(x_n)} \right) \right), \\
        x_{n+1} &= z_n - \dfrac{f(z_n)}{f[y_n, z_n]}\left( 1 - \dfrac{1+f[x_n,w_n]f(z_n)}{f[x_n, w_n]f(w_n)} \right)
    \end{cases}
\end{equation}

除此之外,\cite{chun2012new}\cite{solaiman2019two}\cite{narang2016new}等均提出了不同形式的六阶收敛的方法,虽然它们的计算效率较高,但均具有较为复杂的迭代格式。本文希望能建立一种较为简单的迭代格式对Halley方法进行改造,在避免求二阶导数的基础上实现六阶收敛的方法。\textbf{诚信声明,在我所了解到的文献里没有与本文方法相同的,虽然推导和证明过程均有参考它们,也有相应的引用,但皆为原创和独立完成;如有雷同,是我查找资料不周全,没有恶意抄袭的意思。}

