\documentclass{article}
\usepackage{geometry}
\geometry{margin=3cm, vmargin={3cm}}

% useful packages.
\usepackage{amsfonts}
\usepackage{amsmath}
\usepackage{amssymb}
\usepackage{amsthm}
\usepackage{algorithm}
\usepackage{algorithmicx}
\usepackage{algpseudocode}
\usepackage{appendix}
\usepackage{bm}
\usepackage{ctex}
\usepackage{colortbl}
\usepackage{diagbox}
\usepackage{enumerate}
\usepackage{graphicx}
\usepackage{graphicx}       
\usepackage{hyperref}
\usepackage{multicol}
\usepackage{multirow}
\usepackage{indentfirst}
\usepackage{listings}
\usepackage{booktabs}
\usepackage{tabularx}
\usepackage{fancyhdr}
\usepackage{layout}
\usepackage{xcolor}
\usepackage{xeCJK}
\usepackage{subfigure}
\usepackage{subfiles}
\usepackage{xr}
\usepackage{minted}
\usepackage{fontspec}
\usepackage{makecell}
\usepackage{longtable}

% 取消超链接框框
\hypersetup{hidelinks} 

% 设置算法
\floatname{algorithm}{算法}
\renewcommand{\algorithmicrequire}{\textbf{输入:}}
\renewcommand{\algorithmicensure}{\textbf{输出:}}
% declaration of the parallel for
\algblock{ParFor}{EndParFor}
\algnewcommand\algorithmicparfor{\textbf{parallel for}}
\algnewcommand\algorithmicpardo{\textbf{do}}
\algnewcommand\algorithmicendparfor{\textbf{end parallel for}}
\algrenewtext{ParFor}[1]{\algorithmicparfor\ #1\ \algorithmicpardo}
\algrenewtext{EndParFor}{\algorithmicendparfor}

% 关键词
\providecommand{\keywords}[1]{\vspace{.6em}\textbf{关键词}\quad #1\vspace{.6em}}

% \usepackage{mathpazo} % 设置字体

\usepackage{sysulab}
\setlength{\parindent}{1em}
\renewcommand{\baselinestretch}{1.5}
\newcommand{\ie}{\textit{i.e.}}

% define used colors
\definecolor{GRay}{RGB}{200, 200, 200}

% cancel date demonstration
\date{}

% set experiment report title, used in next.
\newcommand{\rtitle}{六阶收敛的非线性方程组求根方法}
\newcommand{\lecturename}{数值分析}
\newcommand{\instructorname}{黎卫兵/张雨浓}
\newcommand{\sgrade}{2022}
\newcommand{\smajor}{计算机科学与技术}
\newcommand{\sname}{林泽佳}
\newcommand{\sid}{22214373}
\newcommand{\sclass}{学硕2班}

\begin{document}

%%%% DO NOT MODIFY HERE 
\pagestyle{fancy}
\fancyhead{}
\lhead{\sname}
\chead{\rtitle}
\rhead{\today}

% \maketitle
\centerline{\LARGE \rtitle}
\begin{figure}[h]
    \centering
    \includegraphics[width=.2\linewidth]{sysu_logo.png}
\end{figure} 
% set lecture information
\centerline{\textbf{课程: \lecturename} \qquad \textbf{任课教师: \instructorname}}
% set head table
\begin{table}[h]
    \centering
    \begin{tabular}{|c|c|c|c|}
        % \toprule 
        \hline
        \cellcolor{GRay} 年级 & \sgrade & \cellcolor{GRay} 专业 & \smajor \\
        \hline
        \cellcolor{GRay} 学号 & \sid &\cellcolor{GRay} 姓名 & \sname \\
        \hline
        \cellcolor{GRay} 班级 & \sclass &\cellcolor{GRay} 日期 & \today \\
        \hline
        % \bottomrule
    \end{tabular}    
\end{table}

%%%% DO NOT MODIFY ABOVE

\subfile{content/abstract}

\subfile{content/background}

% \subfile{content/related}

\subfile{content/design}

\subfile{content/experiment}

\subfile{content/conclusion}

%
% References will then be sorted and formatted in the correct style
%
\clearpage
\bibliographystyle{unsrt}
\bibliography{ref}

\clearpage
\subfile{content/appendix}

\end{document}
